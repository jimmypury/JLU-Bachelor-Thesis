\chapter{绪论}

\section{研究背景}

\subsection{LaTeX 简介}
\LaTeX 是一种基于 \TeX 排版系统的文档准备系统,广泛应用于学术论文、书籍出版等领域。与传统的 WYSIWYG(所见即所得)编辑器不同,\LaTeX 采用命令驱动的方式进行排版,使得文档的结构和内容得以分离,大大提高了排版的灵活性和可维护性。

\subsection{毕业论文 \LaTeX 模板的必要性}
在撰写本科毕业论文时,学生往往需要遵守学校的格式要求,包括论文结构、字体、页面布局等。手动设置这些格式往往繁琐乏味,而使用 \LaTeX 模板可以大幅提高论文撰写的效率,让学生能够更好地专注于论文内容的撰写。

\section{研究目标}
本文的主要研究目标如下:
\begin{itemize}
\item 设计一款简洁美观、易于使用的 \LaTeX 本科毕业论文模板\cite{brown_online_2023}
\item 使模板能够遵循学校相关规范,包括论文结构、字体、页面布局等\cite{lee_chapter_2022}
\item 支持用户自定义设置,满足不同学生的个性化需求
\item 提高论文撰写效率,让学生能够更好地专注于论文内容的撰写
\end{itemize}

\section{论文结构安排}
本文共分为以下几个章节:
\begin{enumerate}
\item 绪论
\item \LaTeX 模板设计
\item 模板使用示例\cite{smith_neural_2023}
\item 结论与展望
\end{enumerate}

其中,\textit{第二章}介绍了 \LaTeX 模板的设计思路和实现方法;\textbf{第三章}给出了模板的具体使用示例\cite{wang_efficient_2020};最后一章总结了研究成果,并对未来的研究方向进行了展望。(虽然并不是,这里瞎写的)