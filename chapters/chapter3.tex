\chapter{模板使用示例}

\section{算法排版}
\LaTeX 提供了 \verb|algorithm| 和 \verb|algorithmic| 宏包,可以方便地排版算法伪代码。下面是一个示例:

\begin{algorithm}
\caption{二分查找算法}
\label{alg:binary_search}
\begin{algorithmic}[1]
\Procedure{BinarySearch}{$A, x$}
\State $low \gets 1$
\State $high \gets length(A)$
\While{$low \le high$}
\State $mid \gets \lfloor (low + high) / 2 \rfloor$
\If{$A[mid] = x$}
\State \textbf{return} $mid$
\ElsIf{$A[mid] < x$}
\State $low \gets mid + 1$
\Else
\State $high \gets mid - 1$
\EndIf
\EndWhile
\State \textbf{return} $-1$
\EndProcedure
\end{algorithmic}
\end{algorithm}

\section{图表排版}
在论文中,图表是重要的组成部分。\LaTeX 提供了 \verb|figure| 和 \verb|table| 环境来排版图表,并支持图表的引用和交叉引用。

\subsection{图片排版}
下面是一个在 \LaTeX 中插入图片的示例:

\begin{figure}[htbp]
\centering
\includegraphics[width=0.6\textwidth]{example-image.pdf}
\caption{示例图片}
\label{fig:example_image}
\end{figure}

我们可以使用 \verb|\ref| 命令引用 \ref{fig:example_image} 中的图片。

同理,可以使用一些方法插入代码块,如下 \ref{fig:example_code_block} 所示:

\begin{figure}[!htb]
    \centerline{\fbox{\parbox{0.8\linewidth}{
        \code{code test} \\
        \code{1234567890} \\
        \code{qwertyuiop} \\
        \code{ASDFGHJKL}
    }}}
    \caption{\code{code}代码示例}
    \label{fig:example_code_block}
\end{figure}


\subsection{表格排版}
表格是另一个常见的元素,\LaTeX 提供了灵活的表格排版功能。下面是一个示例:

\begin{table}[htbp]
    \centering
\caption{示例表格}
\label{tab:example_table}
\begin{tabular}{c|c|c}
\hline
姓名 & 年龄 & 分数 \\
\hline
张三 & 20 & 90 \\
李四 & 22 & 85 \\
王五 & 19 & 92 \\
\hline
\end{tabular}

\end{table}

我们可以使用 \verb|\ref| 命令引用 \ref{tab:example_table} 中的表格。

\section{结论}
本章通过具体的使用示例,演示了 \LaTeX 模板在算法、图表排版等方面的能力,为读者提供了实践的参考。